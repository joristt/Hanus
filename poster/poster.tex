%----------------------------------------------------------------------------------------
%	PACKAGES AND OTHER DOCUMENT CONFIGURATIONS
%----------------------------------------------------------------------------------------

\documentclass[final,hyperref={pdfpagelabels=false}]{beamer}
\usepackage[orientation=portrait,size=a0,scale=1.7]{beamerposter}
\usetheme{I6pd2} % Use the I6pd2 theme supplied with this template

\usepackage[english]{babel}
\usepackage{amsmath,amsthm,amssymb,latexsym}

\usepackage{minted}
%\usemintedstyle{friendly}
\usemintedstyle{tango}
%\usemintedstyle{monokai}

\graphicspath{{figures/}} % Location of the graphics files

%----------------------------------------------------------------------------------------
%	TITLE SECTION
%----------------------------------------------------------------------------------------

\title{\LARGE \textsc{Hanus: Embedding Janus in Haskell}} % Poster title

\author{\vspace{1cm} Joris ten Tusscher, Joris Burgers, Ivo Gabe de Wolff,\\ Cas van der Rest, Orestis Melkonian\vspace{1cm}} % Authors

\institute{\Large \emph{Faculty of Science, Utrecht University}} % Institution

%----------------------------------------------------------------------------------------
%	FOOTER TEXT
%----------------------------------------------------------------------------------------

\newcommand\footsize{10ex}
\newcommand\leftfoot{
	\begin{minipage}{.5\textwidth}
	\includegraphics[scale=2.3]{uulogo.pdf}
	\end{minipage}
	\vspace{.2cm}
} % Left footer text
\newcommand\rightfoot{
	\begin{minipage}{.5\textwidth}
	\vspace{1cm}
	\large \textbf{https://github.com/joristt/hanus}
	\end{minipage}

%	\vspace{1cm}
} % Right footer text

%----------------------------------------------------------------------------------------
%	SIZES ( 3*sepsize + 2*colsize == 1 )
%----------------------------------------------------------------------------------------
\newcommand\sepsize{.05\textwidth}
\newcommand\colsize{.425\textwidth}

%----------------------------------------------------------------------------------------
%	STYLING
%----------------------------------------------------------------------------------------
\usepackage{times}\usefonttheme{professionalfonts}

%----------------------------------------------------------------------------------------
\newcommand{\code}[2]{
	\begin{center}
		\vspace{.5cm}
		\textsc{\small #1}\\
		\vspace{.5cm}
	\end{center}
	\begin{minipage}{.9\textwidth}
		\inputminted[frame=lines,framesep=1cm,baselinestretch=.8,linenos,fontsize=\footnotesize]
			{haskell}{code/#2.hs}
	\end{minipage}
}
\newcommand{\codeErr}[1]{
	\begin{center}
		\vspace{.5cm}
		\textsc{\small Error}\\
		\vspace{.5cm}
	\end{center}
	\begin{minipage}{.9\textwidth}
		\inputminted[frame=lines,framesep=1cm,baselinestretch=.8,fontsize=\footnotesize]
			{bash}{code/#1_err.hs}
	\end{minipage}
}

\begin{document}

\addtobeamertemplate{block end}{}{\vspace*{4ex}} % White space under blocks

\begin{frame}[t] % The whole poster is enclosed in one beamer frame

\begin{columns}[t] % The whole poster consists of two major columns, each of which can be subdivided further with another \begin{columns} block - the [t] argument aligns each column's content to the top

\begin{column}{\sepsize}\end{column} % Empty spacer column

\begin{column}{\colsize} % The first column

%----------------------------------------------------------------------------------------
%	CONTENT
%----------------------------------------------------------------------------------------

\begin{block}{Introduction}
	\begin{itemize}
		\item \textbf{Janus} is an imperative reversible programming language, meaning that every computation and function can be reversed.
		\vspace{1cm}
 		\item \textbf{Hanus} is an extended implementation of Janus that can be compiled straight to Haskell. Because of this, Hanus contains many awesome Haskell features that are unthinkable in regular Janus!
	\end{itemize}
\end{block}

\begin{block}{Reverse your program}
	\begin{itemize}
	\item The inverse of a program can be computed automatically.
		\code{Division}{divide}
		\vspace{1cm}
		\code{Reverse of division}{divide_reverse}
	\vspace{1cm}
	\item Procedures are called with either the \textit{call} or the \textit{uncall} keyword. The \textit{main} procedure is called automatically.
		\code{Execution}{call_uncall}
	\end{itemize}
\end{block}

\begin{block}{Syntactic Checking}
	\begin{itemize}
	\item By using \textit{QuasiQuotation}, the programmer gets notified of syntactic errors at compile-time!
		\vspace{-.35cm}
		\code{Code}{syntax}
		\vspace{1cm}
		\codeErr{syntax}
	\end{itemize}
\end{block}

%----------------------------------------------------------------------------------------

\end{column} % End of the first column

\begin{column}{\sepsize}\end{column} % Empty spacer column

\begin{column}{\colsize} % The second column
\addtobeamertemplate{block end}{}{\vspace*{-2.4ex}} % White space under blocks

%----------------------------------------------------------------------------------------

\begin{block}{Semantic Checking (Janus side)}
	\begin{itemize}
	\item \textit{Hanus} also reports semantic errors, such as violating Janus-specific constraints for expressions.
		\code{Code}{semJ}
		\vspace{1cm}
		\codeErr{semJ}
	\end{itemize}
\end{block}

\begin{block}{Semantic Checking (Haskell side)}
	\begin{itemize}
	\item Since regular Haskell programs are generated, users also get error messages for \textit{anti-quoted} Haskell expressions.
		\code{Code}{semH}
		\vspace{1cm}
		\codeErr{semH}
	\end{itemize}
\end{block}

\begin{block}{Haskell Power}
	\begin{itemize}
	\item The programmer can add additional operators by defining functions for forward and backward execution.
	\item We can define an operator that works on all Functors:
		\code{Definition}{functor_def}
		\vspace{1cm}
		\code{Usage}{functor_usage}
		\vspace{1cm}
	\item Besides operators, the programmer can also define field and array indexers which allow you to use \texttt{tree.leftChild} and \texttt{array[x]} on the left hand side.
	\end{itemize}
\end{block}

%----------------------------------------------------------------------------------------

\end{column} % End of the second column

\begin{column}{\sepsize}\end{column} % Empty spacer column

\end{columns} % End of all the columns in the poster

\end{frame} % End of the enclosing frame

\end{document}
