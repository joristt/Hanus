%----------------------------------------------------------------------------------------
%	PACKAGES AND OTHER DOCUMENT CONFIGURATIONS
%----------------------------------------------------------------------------------------

\documentclass[final,hyperref={pdfpagelabels=false}]{beamer}
\usepackage[orientation=portrait,size=a0,scale=1.8]{beamerposter}
\usetheme{I6pd2} % Use the I6pd2 theme supplied with this template

\usepackage[english]{babel}
\usepackage{amsmath,amsthm,amssymb,latexsym}

\usepackage{minted}
%\usemintedstyle{friendly}
\usemintedstyle{tango}
%\usemintedstyle{monokai}

\graphicspath{{figures/}} % Location of the graphics files

\usecaptiontemplate{\small\structure{\insertcaptionname~\insertcaptionnumber: }\insertcaption} % A fix for figure numbering

%----------------------------------------------------------------------------------------
%	TITLE SECTION 
%----------------------------------------------------------------------------------------

\title{\LARGE \textsc{Hanus: Embedding Janus in Haskell}} % Poster title

\author{\vspace{1cm} Joris ten Tusscher, Joris Burgers, Ivo Gabe de Wolff,\\ Cas van der Rest, Orestis Melkonian\vspace{1cm}} % Authors

\institute{\Large \emph{Faculty of Science, Utrecht University}} % Institution

%----------------------------------------------------------------------------------------
%	FOOTER TEXT
%----------------------------------------------------------------------------------------

\newcommand\leftfoot{https://github.com/joristt/hanus} % Left footer text
\newcommand\rightfoot{} % Right footer text

%----------------------------------------------------------------------------------------
%	SIZES ( 3*sepsize + 2*colsize == 1 )
%----------------------------------------------------------------------------------------
\newcommand\sepsize{.05\textwidth}
\newcommand\colsize{.425\textwidth}

%----------------------------------------------------------------------------------------
%	STYLING
%----------------------------------------------------------------------------------------
\usepackage{times}\usefonttheme{professionalfonts}
%\usefonttheme[onlymath]{serif}
%\setbeamercolor{block title}{fg=black,bg=orange!70} % Change the block title color

%----------------------------------------------------------------------------------------
\newcommand{\code}[1]{\inputminted[frame=lines,framesep=1cm,baselinestretch=.9,linenos,fontsize=\scriptsize]{haskell}{code/#1.hs}}
\newcommand{\codeErr}[1]{\inputminted[frame=lines,framesep=1cm,baselinestretch=.9,fontsize=\scriptsize]{bash}{code/#1_err.hs}}

\begin{document}

\addtobeamertemplate{block end}{}{\vspace*{4ex}} % White space under blocks

\begin{frame}[t] % The whole poster is enclosed in one beamer frame

\begin{columns}[t] % The whole poster consists of two major columns, each of which can be subdivided further with another \begin{columns} block - the [t] argument aligns each column's content to the top

\begin{column}{\sepsize}\end{column} % Empty spacer column

\begin{column}{\colsize} % The first column

%----------------------------------------------------------------------------------------
%	CONTENT
%----------------------------------------------------------------------------------------
            
\begin{block}{Introduction}
	\begin{itemize}
		\item DSL description
		\item Reversible (Janus)
	\end{itemize}
\end{block}

\begin{block}{Reverse your program}
	\begin{itemize}
		\item Division example
		\item Show inverse side-by-side
	\end{itemize}
\end{block}

\begin{block}{Syntactic Checking}
	\begin{itemize}
		\item By using \textit{QuasiQuotation}, the programmer gets notified of syntactic errors at compile-time!
	\center
	\textsc{\small Code}\\ \vspace{1cm}
	\begin{minipage}{.8\textwidth}
	\code{syntax}
	\end{minipage}
	\vspace{1cm}
	\item \textsc{\small Error}\\ \vspace{1cm}
	\begin{minipage}{.8\textwidth}
	\codeErr{syntax}
	\end{minipage}	
	\end{itemize}
\end{block}

%----------------------------------------------------------------------------------------

\end{column} % End of the first column

\begin{column}{\sepsize}\end{column} % Empty spacer column
 
\begin{column}{\colsize} % The second column

%----------------------------------------------------------------------------------------

\begin{block}{Semantic Checking (Janus side)}
	\begin{itemize}
		\item \textit{Hanus} also reports semantic errors, such as violating Janus-specific constraints for expressions.
		\center
		\textsc{\small Code}\\ \vspace{1cm}
		\begin{minipage}{.8\textwidth}
		\code{semJ}
		\end{minipage}
		\vspace{1cm}
		\item \textsc{\small Error}\\ \vspace{1cm}
		\begin{minipage}{.8\textwidth}
		\codeErr{semJ}
		\end{minipage}
	\end{itemize}
\end{block}

\begin{block}{Semantic Checking (Haskell side)}
	\begin{itemize}
		\item Since regular Haskell programs are generated, users also get error messages for \textit{anti-quoted} Haskell expressions.
		\center
		\textsc{\small Code}\\ \vspace{1cm}
		\begin{minipage}{.8\textwidth}
		\code{semH}
		\end{minipage}
		\vspace{1cm}
		\item \textsc{\small Error}\\ \vspace{1cm}
		\begin{minipage}{.8\textwidth}
		\codeErr{semH}
		\end{minipage}
	\end{itemize}     
\end{block}

\begin{block}{Haskell Power}
	\begin{itemize}
		\item The programmer can add additional operators by defining functions for forward and backward execution.
		\item We can define an operator that works on all Functors:
		\begin{center}
		\textsc{\small Definition}\\ \vspace{1cm}
		\begin{minipage}{.8\textwidth}
			\code{functor_def}
		\end{minipage}
		\vspace{1cm}
		\item \textsc{\small Usage}\\ \vspace{1cm}
		\begin{minipage}{.8\textwidth}
			\code{functor_usage}
		\end{minipage}
		\end{center}
		\vspace{1cm}
		\item Besides operators, the programmer can also define field and array indexers which allow you to use \texttt{tree.leftChild} and \texttt{array[x]} on the left hand side.
	\end{itemize}
\end{block}

%----------------------------------------------------------------------------------------

\end{column} % End of the second column

\begin{column}{\sepsize}\end{column} % Empty spacer column

\end{columns} % End of all the columns in the poster

\end{frame} % End of the enclosing frame

\end{document}