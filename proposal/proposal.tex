\documentclass[12pt,a4paper]{article}

% Tables
\usepackage{multicol}
\usepackage{multirow}

% Colors
\usepackage{xcolor, color, colortbl}
\colorlet{gray}{gray!70}
\colorlet{green}{green!50}

% Links
\usepackage{hyperref}
\definecolor{linkcolour}{rgb}{0,0.2,0.6}
\hypersetup{colorlinks,breaklinks,urlcolor=linkcolour,linkcolor=linkcolour}

\title{\textbf{Project Proposal \\ \small{Concepts of Program Design}}}
\author{\small{Joris ten Tusscher, Joris Burgers, Ivo-Gabe de Wolff, Cas van der Rest, Orestis Melkonian}}
\date{}

\begin{document}
\maketitle

\section{Background}
\begin{itemize}
	\item{Reversible Computation}
	\item{Janus}
	\item{other languages}
	\item{a bit of theory}
\end{itemize}

\section{Problem}
	\subsection{Experiment}
	
	\subsection{Extensions}
	\begin{itemize}
		\item{Data structures (trees/sets/maps/ADTs)}
		\item{Prelude}
		\item{Syntactic sugar}
	\end{itemize}
	
	\subsection{Applications}
	\begin{itemize}
		\item{Pathfinding algorithms}
        \item{Encoding-Decoding}
        \item{Low-level bit manipulation (hamming error correction)}
        \item{[Optional] Reversible Debugger}
	\end{itemize}
	
	\subsection{Comparison with Existing Implementations}
    \subsubsection{Benchmarking}
    We intend to compare the performance of our embedded DSL to existing implementations of the Janus language. We will do so by implementing the same algorithms in both our DSL and standard Janus, and comparing the time it takes the different implementations to compile and run the programs, as well as memory usage.
    \subsubsection{Other Metrics}
    Since our main goal is to improve the usability of Janus, it is hard to come up with metrics that provide a useful insight in the quality of our result. Certainly, performance alone will not be enough to capture the differences between the embedding and existing implementations. In order to attempt to measure the quality of our result beyond raw performance, we will compare our results with existing implementations with respect to the following aspects as well. 
	\begin{itemize}
		\item{LOC when implementing a certain algorithm}
		\item{The amount of language constructs and abstractions at a programmer's disposal when solving problems}
        \item{The ability to use patterns/constructs that are generally considered as idiomatic or elegant within the programming community}
	\end{itemize}
Admittedly, most of the above comes down to opinion. We think that it is useful to think about these aspects when considering our results nonetheless. 
	
	\subsection{Formal verification}
	\begin{itemize}
		\item{Reversibility}
		\item{Verifying pre/post-conditions}
		\item{r-Turing Completeness}
	\end{itemize}

\section{Methodology}
\begin{itemize}
	\item{Template Haskell \footnote{\url{https://wiki.haskell.org/Template_Haskell}}}
	\begin{itemize}
		\item{Embedding Janus}
		\item{Compile-time guarantees}
		\begin{itemize}
			\item{Type-checking}
			\item{Variable usage}
		\end{itemize}
		
	\end{itemize}
	
	\item{GHC Profiling}
	\begin{itemize}
		\item{Criterion package \footnote{\url{http://hackage.haskell.org/package/criterion})}}
	\end{itemize}
	
	\item{Liquid Haskell \footnote{\url{https://ucsd-progsys.github.io/liquidhaskell-blog/})}}
	\begin{itemize}
		\item{Theorem proving}
	\end{itemize}
\end{itemize}

\section{Planning}

\setlength{\tabcolsep}{20pt}
\renewcommand{\arraystretch}{1.6}
\begin{center}
\begin{tabular}{cl}
\multirow{2}{*}{\textsc{week 1}} &
	\textbf{Task 1:} Orestis,... \\ {} &
	\textbf{Task 2:} Cas,... \\ \hline
\multirow{2}{*}{\textsc{week 2}} &
	\textbf{Task 3:} Joris1, Joris2,... \\ {} &
\textbf{Task 4:} Ivo,... \\ \hline
\multirow{2}{*}{\textsc{week 3}} &
	\textbf{Task 1:} Orestis,... \\ {} &
	\textbf{Task 2:} Cas,... \\
\rowcolor{green} \multicolumn{2}{c}{\textsc{Progress Report}} \\
\multirow{2}{*}{\textsc{week 4}} &
	\textbf{Task 3:} Joris1, Joris2,... \\ {} &
	\textbf{Task 4:} Ivo,... \\ \hline
\multirow{2}{*}{\textsc{week 5}} &
	\textbf{Task 1:} Orestis,... \\ {} &
	\textbf{Task 2:} Cas,... \\ \hline
\multirow{2}{*}{\textsc{week 6}} &
	\textbf{Task 3:} Joris1, Joris2,... \\ {} &
	\textbf{Task 4:} Ivo,... \\
\rowcolor{green} \multicolumn{2}{c}{\textsc{Project Submission}}
\end{tabular}
\end{center}
\end{document}
