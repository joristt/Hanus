\documentclass[12pt,a4paper]{article}

% Tables
\usepackage{multicol}
\usepackage{multirow}

% Colors
\usepackage{xcolor, color, colortbl}
\colorlet{gray}{gray!70}
\colorlet{green}{green!50}

% Links
\usepackage{hyperref}
\definecolor{linkcolour}{rgb}{0,0.2,0.6}
\hypersetup{colorlinks,breaklinks,urlcolor=linkcolour,linkcolor=linkcolour}

\title{\textbf{Project Proposal \\ \small{Concepts of Program Design}}}
\author{\small{Joris ten Tusscher, Joris Burgers, Ivo-Gabe de Wolff, Cas van der Rest, Orestis Melkonian}}
\date{}

\begin{document}
\maketitle

\section{Background}
\begin{itemize}
	\item{Reversible Computation}
	\item{other languages}
	\item{a bit of theory}
\end{itemize}

\section{Problem}
	\subsection{Experiment}
	
	\subsection{Janus}
	We want to implement the reversible programming language Janus as an DSL in Haskell. The syntax and semantics given by Lutz en Derby \cite{r:lutz-derby1982} will be used. There will however be a few changes. There will be no input and output in the DSL. Therefore, the READ and WRITE statements will not be valid statements. Besides this limitation, the DSL will allow arbitrary Haskell expressions to the right-hand side of Janus statements, enabling programmers to reuse Haskell code. This freedom of accepting Haskell code will not limit the reversibility of the DSL, since the semantics for Janus require statements to be reversible, but not right-hand expressions. For instance, the statement $x += f$ where $f$ is any valid Haskell expression. As long as $f$ does not depend on $x$, the statement can be inverted by the statement $x -= f$.
	
	\subsection{Extensions}
	\begin{itemize}
		\item{Data structures (trees/sets/maps/ADTs)}
		\item{Prelude}
		\item{Syntactic sugar}
	\end{itemize}
	
	\subsection{Applications}
	\begin{itemize}
		\item{Path finding algorithms}
        \item{Encoding-Decoding}
        \item{Low-level bit manipulation (hamming error correction)}
        \item{[Optional] Reversible Debugger}
	\end{itemize}
	
	\subsection{Comparison with Existing Implementations}
    \subsubsection{Benchmarking}
    We intend to compare the performance of our embedded DSL to existing implementations of the Janus language. We will do so by implementing the same algorithms in both our DSL and standard Janus, and comparing the time it takes the different implementations to compile and run the programs, as well as memory usage.
    \subsubsection{Other Metrics}
    Since our main goal is to improve the usability of Janus, it is hard to come up with metrics that provide a useful insight in the quality of our result. Certainly, performance alone will not be enough to capture the differences between the embedding and existing implementations. In order to attempt to measure the quality of our result beyond raw performance, we will compare our results with existing implementations with respect to the following aspects as well. 
	\begin{itemize}
		\item{LOC when implementing a certain algorithm}
		\item{The amount of language constructs and abstractions at a programmer's disposal when solving problems}
        \item{The ability to use patterns/constructs that are generally considered as idiomatic or elegant within the programming community}
	\end{itemize}
Admittedly, most of the above comes down to opinion. We think that it is useful to think about these aspects when considering our results nonetheless. 
	
	\subsection{Formal verification}
	\begin{itemize}
		\item{Reversibility}
		\item{Verifying pre/post-conditions}
		\item{r-Turing Completeness}
	\end{itemize}

\section{Methodology}
\subsection{Risk and contingency plans}
There are a number of risks to consider when creating a DSL, which are specified below with their respective contingency plans.
\begin{itemize}
\item \textbf{Arbitrary types} The plan is to implement arbitrary types in the DSL. It is expected that implementing arbitrary types will take a long time and therefore, the risk exists that the implementation can not be completed in the given timespan. The contingency plan, if this goal takes too much time, is to reduce arbitrary types to some special types.
\item \textbf{Full verification} The goal is to completely verify the implementation of the DSL. To our knowledge, there is, to our knowledge, no verified implementation of a reversible language. Therefore, we are aiming at providing a fully verified implementation. If this full verification is not possible, the contingency plan is to at least verify the basic, non-extended version of Janus. 
\item \textbf{Impossible applications} It is a possibility that the applications that will be implemented in the DSL are impossible to implement in a reversible language. If there is an application where there is no suitable variant found that can be implemented in a reversible language, that specific application will be ignored or a suitable replacement will be looked for. 
\item \textbf{Impossible use of libraries} There is the risk that one of the proposed libraries we plan to use to implement the DSL turns out not to be suitable. This may be the case because the library lacks some features that are necessary for the implementation of the DSL. If this is the case, a replacement will be looked for. If no suitable replacement can be found, the functionality will either be implemented in another way without the use of a library or will be removed from the specification.
\end{itemize} 


\begin{itemize}
	\item{Template Haskell \footnote{\url{https://wiki.haskell.org/Template_Haskell}}}
	\begin{itemize}
		\item{Embedding Janus}
		\item{Compile-time guarantees}
		\begin{itemize}
			\item{Type-checking}
			\item{Variable usage}
		\end{itemize}
		
	\end{itemize}
	
	\item{GHC Profiling}
	\begin{itemize}
		\item{Criterion package \footnote{\url{http://hackage.haskell.org/package/criterion})}}
	\end{itemize}
	
	\item{Liquid Haskell \footnote{\url{https://ucsd-progsys.github.io/liquidhaskell-blog/})}}
	\begin{itemize}
		\item{Theorem proving}
	\end{itemize}
\end{itemize}

\section{Planning}

\setlength{\tabcolsep}{20pt}
\renewcommand{\arraystretch}{1.6}
\begin{center}
\begin{tabular}{cl}
\multirow{2}{*}{\textsc{week 1}} &
	\textbf{Task 1:} Orestis,... \\ {} &
	\textbf{Task 2:} Cas,... \\ \hline
\multirow{2}{*}{\textsc{week 2}} &
	\textbf{Task 3:} Joris1, Joris2,... \\ {} &
\textbf{Task 4:} Ivo,... \\ \hline
\multirow{2}{*}{\textsc{week 3}} &
	\textbf{Task 1:} Orestis,... \\ {} &
	\textbf{Task 2:} Cas,... \\
\rowcolor{green} \multicolumn{2}{c}{\textsc{Progress Report}} \\
\multirow{2}{*}{\textsc{week 4}} &
	\textbf{Task 3:} Joris1, Joris2,... \\ {} &
	\textbf{Task 4:} Ivo,... \\ \hline
\multirow{2}{*}{\textsc{week 5}} &
	\textbf{Task 1:} Orestis,... \\ {} &
	\textbf{Task 2:} Cas,... \\ \hline
\multirow{2}{*}{\textsc{week 6}} &
	\textbf{Task 3:} Joris1, Joris2,... \\ {} &
	\textbf{Task 4:} Ivo,... \\
\rowcolor{green} \multicolumn{2}{c}{\textsc{Project Submission}}
\end{tabular}
\end{center}
\end{document}
