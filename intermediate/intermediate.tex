\documentclass[12pt,a4paper]{article}

\usepackage{minted}

% Tables
\usepackage{multicol}
\usepackage{multirow}
\usepackage{csquotes}
\usepackage{fullpage}

% Colors
\usepackage{xcolor, color, colortbl}
\colorlet{gray}{gray!70}
\colorlet{green}{green!50}

% Links
\usepackage{hyperref}
\definecolor{linkcolour}{rgb}{0,0.2,0.6}
\hypersetup{colorlinks,breaklinks,urlcolor=linkcolour,linkcolor=linkcolour}

\title{\textbf{Intermediate report \\ \small{Concepts of Program Design}}}
\author{\small{Joris ten Tusscher, Joris Burgers, Ivo Gabe de Wolff, Cas van der Rest, Orestis Melkonian}}
\date{}

% Macros
\newcommand{\site}[1]{\footnote{\url{#1}}}
\newcommand{\code}[1]{\mintinline{bash}{#1}}

\begin{document}
\maketitle


\section{Problem}
    \subsection{Experiment}
    The main research question that we want to be able to answer when we have conducted our experiment is:
    
    \begin{displayquote}
    \textit{can we improve the usability of Janus by implementing it as a DSL embedded in Haskell?}
    \end{displayquote}
    
    \noindent In order to be able to answer this question, we will attempt to improve Janus in several, mainly by incorporating various elements of the host language (Haskell) in the DSL, as well as the addition of certain language constructs. 
    We will implement certain algorithms in both our DSL and standard Janus and compare the implementations with respect various metrics, such as performance and readability. 
    
    \subsection{Janus}
    We want to implement the reversible programming language Janus as an DSL in Haskell. The syntax and semantics given by Lutz en Derby \cite{lutz82} will be used. There will however be a few changes. There will be no input and output in the DSL. Therefore, the READ and WRITE statements will not be valid statements
    
    \subsection{Extensions}
    We have various extensions, mainly focused on improving the usability of our DSL, that we would like to implement if time permits.
    
\begin{itemize}
\item 
Local variable blocks
\item
Function arguments
\item
Arbitrary types
\item
Compile-time guarantees on type safety and reversibility. 
\end{itemize}
    
    Some Janus features, such as the stack and arrays will not be implemented directly into the DSL, but are available through predefined functions in the host language.
\section{Methodology}
\subsection{DSL Embedding}
\label{subsec:template-haskell}
In order to embed Janus as a DSL in Haskell, we are going to use the \textit{TemplateHaskell}(TH)\cite{sheard02} and \textit{QuasiQuotation}\cite{mainland07} GHC extensions, which extend Haskell with compile-time meta-programming capabilities. This will allow us to perform static checking (i.e. syntactic and semantic checking), code generation and embedding Haskell expressions in Janus at compile-time.

On the static checking side of things, TH will allow us to perform the following at compile-time:
\begin{itemize}
    \item{Parsing}
    \item{Checking programs are well-typed}
    \item{Proper variable usage}
    \item{Embedding of Haskell expressions}
\end{itemize}

Specifically, Janus programs will be written as \textit{quasi-quote} strings inside a Haskell source file, which will eventually transfer control to a \textit{quasi quoter}, that will parse and statically check desired properties of the given program, as well as compile the domain-specific syntax to Haskell code.

For instance, Janus statements of the form \textbf{id} $\langle\textsc{reversibleOp}\rangle$ \textbf{expr} require that the identifier on the left-hand side does not occur in the expression on the right-hand side. This is easily achieved in TH via the process of \textit{reification}, which allows us to query compile-time information (e.g. a variable's type and identifier) while running our meta-programs.

In addition to compile-time guarantees, the embedded Janus language will also inherit powerful features, already existent in Haskell (e.g. modularity via Haskell's modules).

\subsection{Expected outcome}
At the end of the project, we expect to have a working DSL, that supports all features from the initial Janus version \cite{lutz82}. We expect to have implemented local variables and function arguments from the extended Janus version \cite{yokoyama10} and we hope to have support for arbitrary types. If we cannot implement the latter, we do not consider the project to be failed, but this makes the DSL less usable. Furthermore, the implementation should have compile-time guarantees on the reversibility of programs as well as compile-time type safety. 

\subsection{Planning}
\setlength{\tabcolsep}{20pt}
\renewcommand{\arraystretch}{1.6}
\begin{center}
\begin{tabular}{cl}
\multirow{2}{*}{\textsc{week 1}} &
    \textbf{Defining the AST:} Ivo, JorisB  \\ &
    \textbf{Parser:} Ivo, JorisB \\ &
    \textbf{Evaluation:} Cas, JorisT\\ &
    \textbf{Finding suitable algorithms:} JorisT\\ &
    \textbf{Setup Template Haskell:} Orestis \\ \hline
\multirow{2}{*}{\textsc{week 2}} &
    \textbf{Evaluation:} Cas, JorisT \\ {} &
    \textbf{Local variables:} Cas, JorisB \\ &
    \textbf{Function parameters:}  Cas, JorisB\\ &
    \textbf{Semantic checker:} Orestis  \\ &
    \textbf{Arbitrary types:}  Ivo, JorisT\\ \hline
\multirow{2}{*}{\textsc{week 3}} &
    \textbf{Local variables:} Cas, JorisB \\ {} &
    \textbf{Function parameters:}  Cas, JorisB\\ &
    \textbf{Initial benchmarking setup:} Orestis \\ &
    \textbf{Arbitrary types:}  Ivo, JorisT \\ &
    \textbf{Intermediate report:} Everybody  \\
\rowcolor{green} \multicolumn{2}{c}{\textsc{Progress Report}} \\
\multirow{2}{*}{\textsc{week 4}} &
    \textbf{Presentation}  \\ {} &
    \textbf{Implement algorithms}  \\ &
    \textbf{Arbitrary types}  \\
\rowcolor{green} \multicolumn{2}{c}{\textsc{Presentation}} \\
\multirow{2}{*}{\textsc{week 5}} &
    \textbf{Implement algorithms}  \\ & 
    \textbf{Analysis of benchmarking results}  \\ {} &
    \textbf{Benchmarking} \\ \hline
\multirow{2}{*}{\textsc{week 6}} &
    \textbf{Final report}  \\ & \\
    
\rowcolor{green} \multicolumn{2}{c}{\textsc{Project Submission}}
\end{tabular}
\end{center}

\newpage

\section{Intermediate results}

\subsection{Parser}
For the DSL, a parser has to be implemented that can parse a string to a Janus abstract syntax tree. This parser is written using the \textit{uu-parsinglib}, a Haskell package that uses error correction. Error correction means that the parser will try to correct the input if necessary. Besides the Janus syntax that is parsed with the \textit{uu-parsinglib}, the DSL needs to support arbitrary Haskell expressions and Haskell types. The parsing of these pieces will be done by the Haskell package \textit{haskell-src-meta}. This package provides functions that will take a \textit{string} as an argument and return a Template Haskell \textit{Exp} for an expression or a Template Haskell \textit{Type} for the types. These types and expressions can cause some problems, because the function that parses an \textit{Exp} or a \textit{Type} expects a string as parameter. This means that the exact string that contains the expression or type needs to be determined. This is problematic, as there is no easy way to determine where the type or the expression ends. The way it is implemented in the parser is that it is known which character or characters are expected immediately after the expression or type. The parser parses until it finds the first terminating character that it can parse. If this is a valid parsing, the parser continues to the second character in the string that is a terminating character, continuing until the parser fails. If that happens, it backs up one step and then returns the parsed expression. An example is the Janus statement:
    \begin{displayquote}
        if((1+2)==3)
    \end{displayquote}
In this example, the Haskell statement is the string \textit{(1+2)==3}. The if statement in Janus needs parentheses around the expression. The parser knows that there should be a Haskell expression just after the first parenthesis. It starts by parsing the string \textit{(1+2} because it tries to parse without parsing the first closing parenthesis. This fails, so it continues with \textit{(1+2)==3}, which is a valid Haskell expression. This expression is then used. This way, all possible Haskell expressions are supported by the parser.

\subsection{Evaluation}
In order to actually run programs written in the DSL, we need some kind of evaluation mechanism. Since the goals of this project include added compile-time guarantees for Janus with respect to type safety, we used Template Haskell for this. The basic approach is to convert a Janus AST to an equivalent Template Haskell representation. Running the resulting program is trivially done by splicing this object into another file. This allows us to use Haskell's type system to enforce type safety at compile-time, as well as enabling the usage of arbitrary types within the DSL.

Currently, we have implemented an evaluator that allows us to convert most elements of a Janus AST to an equivalent Template Haskell representation. The evaluator enforces compile-time type safety on the provided AST and provides support for arbitrary types. Although we consider this to be decent progress, there are still some elements from Janus that have to be implemented in the evaluator:
\begin{itemize}
\item 
While-loops
\item
Function arguments
\item
Local variables
\item
Proper scoping for local and global variables. 
\end{itemize}
We think that we can reasonably implement these features in the remaining time. 

\subsection{Semantic Checking}
Although most of the semantic checks we wish to perform on Janus programs are performed by our beloved host language, we also implemented the following Janus-specific checks:
\begin{enumerate}
\item There should be at least one \code{main} procedure declared.
\item Expressions on the right-hand side of assigment statements must not use any of the variables on the left-hand side, such as \code{x += x + 1}.
\end{enumerate}
As we already have the privilege of staged compilation, we run the aforementioned checks during compilation time, thus informing the end-user much earlier on in the development process.

Moreover, we might add more checks, as we discover possibilities along the way. For instance, we could detect certain common errors where the user has provided guards at the end of an \code{if}-statement, which cannot determine which of the two branches was taken, hence rendering the statement impossible to run in reverse.

\section{Timeline}
Everything, more or less, went according to our original plan, except a minor setback last week, which we will, nonetheless, cover during the forthcoming Christmas break.

Although we had originally planned to add features in an incremental fashion (e.g. having a fully-working implementation of basic Janus at first and later allow embedded Haskell expressions), we followed a more monolithic approach and started with all desired features in mind. We think this proved to be more efficient and mitigated the unnecessary overhead of refactoring a major part of the codebase with each additional feature.

On the one hand, we got a bit delayed with the implementation of the evaluation step (i.e. the generation of the Haskell code that corresponds to the forward and backward version of a Janus program), since it proved to be more difficult and contain more subtle issues, than we had anticipated.

On the other hand, we are a week ahead of schedule on allowing usage of arbitrary types, which we can contribute to our monolithic approach to feature inclusion. We expect this co-delay to cancel out the delay mentioned above.

Finally, we have postponed the benchmarking setup and survey of suitable algorithms one week, since we are more concerned about having a stable DSL, before continuing with the analysis/benchmarking part.

\newpage

\bibliographystyle{ieeetr}
\bibliography{sources}

\end{document}